% !TEX root = ./Basilisk-celestialTwoBodyPoint-20190311.tex


\section{Test Description and Success Criteria}
The mathematics in this module are straight forward and can be tested in a series of input and output evaluation tests.

\subsection{Test 1}
The first test does a single-body celestial point. It places a spacecraft around Earth on an orbit with parameters given in \ref{tab:orbitParams}. 
\begin{table}[htbp]
	\caption{Spacecraft Orbital Paramters}
	\label{tab:orbitParams}
	\centering \fontsize{10}{10}\selectfont
	\begin{tabular}{ c | c } % Column formatting, 
		\hline\hline
		\textbf{Orbital Parameter}  & \textbf{Value}  \\ 
		\hline
		$a$    		        	     &  $2.8 R_{\text{earth}}	$   \\ 
		$e$           &  $0  $ \\ 
		$i$  & $0 \dg $\\ 
		$\Omega$  & $0\dg  $\\ 
		$\omega$  & $0\dg$\\ 
		$f$  & $60 \dg $\\ 
		\hline\hline
	\end{tabular}
\end{table}
The earth position and velocity vectors are both set to the zero vectors.

\subsection{Test 2}
The second test does the 2-body celestial point. It uses the same parameters at above and sets a second planet with zero velocity and position vector:

\begin{equation}
\leftexp{N}{\bm r_2} = \begin{bmatrix}500 & 500 &500 \end{bmatrix}^T (\text{km})
\end{equation}

\section{Test Parameters}
For each of these two tests, the tested parameters are listed Table~\ref{tab:errortol}. 
\begin{table}[htbp]
	\caption{Error tolerance for each test.}
	\label{tab:errortol}
	\centering \fontsize{10}{10}\selectfont
	\begin{tabular}{ c | c } % Column formatting, 
		\hline\hline
		\textbf{Output Value Tested}  & \textbf{Tolerated Error}  \\ 
		\hline
		$\bm\sigma_{R/N}$     		        	     & \input{AutoTeX/toleranceValue}  \\ 
		$\leftexp{N}{\bm\omega_{R/N}}$            &  \input{AutoTeX/toleranceValue} \\ 
		$\leftexp{N}{\dot{\bm \omega}_{R/N}}$   & \input{AutoTeX/toleranceValue}\\ 
		\hline\hline
	\end{tabular}
\end{table}




\section{Test Results}

All of the tests passed:
\begin{table}[H]
	\caption{Test results}
	\label{tab:results}
	\centering \fontsize{10}{10}\selectfont
	\begin{tabular}{c | c  } % Column formatting, 
		\hline\hline
		\textbf{Check} 						  		&\textbf{Pass/Fail} \\ 
		\hline
	   1.1	   			& \input{AutoTex/passFail11} \\ 
	   1.2	   			& \input{AutoTex/passFail12} \\ 
	   1.3	   			& \input{AutoTex/passFail13} \\ 
	   2.1	   			& \input{AutoTex/passFail21} \\ 
	   2.2	   			& \input{AutoTex/passFail22} \\ 
	   2.3	   			& \input{AutoTex/passFail23} \\ 
	   \hline\hline
	\end{tabular}
\end{table}



