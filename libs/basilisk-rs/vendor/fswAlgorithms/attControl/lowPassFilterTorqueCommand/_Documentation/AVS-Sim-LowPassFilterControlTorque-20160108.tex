\documentclass[]{BasiliskReportMemo}

\usepackage{cite}
\usepackage{AVS}
\usepackage{float} %use [H] to keep tables where you put them
\usepackage{array} %easy control of text in tables
\usepackage{graphicx}
\bibliographystyle{plain}


\newcommand{\ModuleName}{lowPassFilterControlTorque}
\newcommand{\status}{Status}

\newcommand{\submiterInstitute}{Autonomous Vehicle Simulation (AVS) Laboratory,\\ University of Colorado}

\begin{document}

\makeCover



%
%	enter the revision documentation here
%	to add more lines, copy the table entry and the \hline, and paste after the current entry.
%
\pagestyle{empty}
{\renewcommand{\arraystretch}{2}
\noindent
\begin{longtable}{|p{0.5in}|p{4.5in}|p{1.14in}|}
\hline
{\bfseries Rev}: & {\bfseries Change Description} & {\bfseries By} \\
\hline
Draft & Initial Documentation & H. Schaub \\
\hline

\end{longtable}
}

\newpage
\setcounter{page}{1}
\pagestyle{fancy}

\tableofcontents
~\\ \hrule ~\\

\section{Introduction}
A low-pass filter module provides the ability to apply a frequency based filter to the ADCS control torque vector $\bm L_{r}$.  The module has the ability to be individually reset, separate from any reset on the ADCS control modules.  The cut-off frequency is given by $\omega_{c}$, while the filter time step is given by $h$.  

\section{Initialization}
Prior to using the module, the filter time step and 1st order filter frequency cut-off value must be set.
\begin{gather*}
{\tt Config->h} \\
 {\tt Config->wc} 
\end{gather*}

\section{Algorithm}
Since the shown mappings between the Laplace domain and the $Z$-domain 
are approximate, some frequency warping will occur.  If a continuous 
filter design has a critical frequency $\omega_{c}$, then the digital 
implementation might have a slightly different critical frequency.  
Franklin in Reference~\citenum{franklin1} compares the continuous and 
digital filter performances by studying the half-power point.  This 
leads to the following relationship between the continuous time 
critical filter frequency $\omega_{c}$ and the digital filter 
frequency $\hat\omega$:
\begin{equation}
	\label{eq:wa}
	\tan \left(\frac{ w_{c} h}{2}\right) = \frac{\hat\omega h}{2}
\end{equation}
where $h=1/f$ is the digital sample time.  Note that $\hat\omega \approx 
\omega_{c}$ if the sample frequency is much higher than the critical 
filter frequency. 

The first-order digital filter formula is given by:
\begin{equation}
		y_{k} = \frac{1}{2+h \omega_{c}}
		\Big[
		y_{k-1} (2-h \omega_{c})  + h \omega_{c} (x_{k} + x_{k-1})
		\Big]
\end{equation}
where $x_{k}$ is the current filter input, and $y_{k}$ is the filtered output.


\section{Output}
The filter module outputs the standard ADCS control torque output structure.  If the original and filtered ADCS control is to be tracked, then the messages should be given unique names.


\bibliographystyle{unsrt}
\bibliography{references}




\end{document}
