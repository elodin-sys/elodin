% !TEX root = ./Basilisk-inertialUKF-20190402.tex


\section{Module Functions}

\begin{itemize}
    \item \textbf{relativeOD UKF Time Update: } Performs the filter time update as defined in the baseline algorithm
    \item \textbf{relativeOD UKF Meas Update: } Performs the filter measurement update as defined in the baseline algorithm
    \item \textbf{relativeOD UKF Two Body Dynamics: } Provides the function used for integration and incorporates all the known dynamics
    \item \textbf{relativeOD UKF Meas Model: } Predicts the measurements given current state and measurement model $\bm G$
    \item \textbf{relativeOD State Prop: } Integrates the state given the $\bm F$ dynamics of the system
    \item \textbf{relativeOD Clean Update: } Returns filter to a previous state in the case of a bad computation
    \end{itemize}

\section{Module Assumptions and Limitations}

The assumptions of this module are all tied in to the underling assumptions and limitations to a working filter. 
In order for a proper convergence of the filter, the dynamics need to be representative of the actual spacecraft perturbations. In this module, the dynamics implemented in the filter are currently just two-body dynamics. Many more perturbations could be added in the future. 

Depending on the tuning of the filter (process noise value and measurement noise value), the robustness of the solution will be weighed against it's precision. 
The number of measurements and the frequency of their availability also influences the general performance. 

