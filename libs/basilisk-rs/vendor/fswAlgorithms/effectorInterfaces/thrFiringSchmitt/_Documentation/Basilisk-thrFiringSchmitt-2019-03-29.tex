% README file for moduleDocumentationTemplate TeX template.
% This template should be used to document all Basilisk modules.
% Updated 20170711 - S. Carnahan
%
%-Copy the contents of this folder to your own _Documentation folder
%
%-Rename the Basilisk-moduleDocumentationTemplate.tex appropriately
%
% All edits should be made in one of:
% sec_modelAssumptionsLimitations.tex
% sec_modelDescription.tex
% sec_modelFunctions.tex
% sec_revisionTable.tex
% sec_testDescription.tex
% sec_testParameters.tex
% sec_testResults.tex
% sec_user_guide.tex
%
% NOTE: if the TeX document is reading in auto-generated TeX snippets from the AutoTeX folder, then
%            pytest must first be run for the unit test of this module.  This process creates the required unit test results 
%.           that are read into this document.  
%
%-Some rules about referencing within the document:
%1. If writing the user guide, assume the module description is present
%2. If writing the validation section, assume the module features section is present
%3. Make no other assumptions about any sections being present. This allow for sections of the document to be used elsewhere without breaking.

%In order to import some of these sections into a document in a different directory:
%\usepackage{import}
%Then, the sections are called with \subimport{relative path}{file} in order to \input{file} using the right relative path.
%\import{full path}{file} can also be used if absolute paths are preferred over relative paths.

%%%%%%%%%%%%%%%%%%%%%%%%%%%%%%%%%%%%%%%%%%%%%%%%%




\documentclass[]{BasiliskReportMemo}

\usepackage{cite}
\usepackage{AVS}
\usepackage{float} %use [H] to keep tables where you put them
\usepackage{array} %easy control of text in tables
\usepackage{graphicx}
\bibliographystyle{plain}


\newcommand{\submiterInstitute}{Autonomous Vehicle Simulation (AVS) Laboratory,\\ University of Colorado}


\newcommand{\ModuleName}{thrFiringSchmitt}
\newcommand{\subject}{Schmitt Trigger Thruster Firing Logic Module}
\newcommand{\status}{Public Release}
\newcommand{\preparer}{H. Schaub}
\newcommand{\summary}{A Schmitt trigger logic is implemented to map a desired thruster force value into a thruster on command time. The module reads in the attitude control thruster force values for both on- and off-pulsing scenarios, and then maps this into a time which specifies how long a thruster should be on.  The thruster configuration data is read in through a separate input message in the reset method.  The Schmitt trigger allows for an upper and lower bound where the thruster is either turned on or off.}

\begin{document}

\makeCover

%
%	enter the revision documentation here
%	to add more lines, copy the table entry and the \hline, and paste after the current entry.
%
\pagestyle{empty}
{\renewcommand{\arraystretch}{2}
\noindent
\begin{longtable}{|p{0.5in}|p{3.5in}|p{1.07in}|p{0.9in}|}
\hline
{\bfseries Rev} & {\bfseries Change Description} & {\bfseries By}& {\bfseries Date} \\
\hline
1.0 & First Release & H. Schaub & 2019-03-29\\
\hline

\end{longtable}
}



\newpage
\setcounter{page}{1}
\pagestyle{fancy}

\tableofcontents %Autogenerate the table of contents
~\\ \hrule ~\\ %Makes the line under table of contents









	
% !TEX root = ./Basilisk-ephemDifference-2019-03-27.tex



\section{Model Description}
The purpose of this module is to rebase ephemeris position and velocity vectors relative to another celestial object.  All the input messages are assumed to have the vectors taken with respect to the same coordinate frame.

Let $\bm r_{P_{i}/N}$ be the position vector of the $i^{\text{th}}$ ephemeris, while $\bm r_{B/N}$ is the position vector of the bases ephemeris message.  The velocity vectors $\bm v_{P_{i}/N}$ and $\bm v_{B/N}$ are defined similarly.  Taking all vectors components with respect to a command inertial frame $\cal N$, the output is computed using
\begin{align}
	\leftexp{N}{\bm r}_{P_{i}/B} &= \leftexp{N}{\bm r}_{P_{i}/N} - \leftexp{N}{\bm r}_{B/N}
	\\
	\leftexp{N}{\bm v}_{P_{i}/B} &= \leftexp{N}{\bm v}_{P_{i}/N} - \leftexp{N}{\bm v}_{B/N}
\end{align}

The time tag of the output message is copied from the corresponding input message, not the base ephemeris message.

The number of input messages to consider is determined by searching the {\tt ephInMsg} and {\tt ephOutMsg} names and finding the first zero string where either name was not set.

 %This section includes mathematical models, code description, etc.

% !TEX root = ./Basilisk-ephemDifference-2019-03-27.tex


\section{Module Functions}
\begin{itemize}
	\item \textbf{Variable input messages}: The user can specify up to {\tt MAX\_NUM\_CHANGE\_BODIES} input messages {\tt ephInMsg}.
	\item \textbf{Number of messages}: The first zero message string terminates the loop and sets the number of incoming and outgoing messages
\end{itemize}

\section{Module Assumptions and Limitations}
The module assumes all vectors are provided with respect to a common coordinate frame.  

Only the first $n$ non-empty string names are used to subscribe to the ephemeris input messages.  The user must setup the equivalent output messages.   %This includes a concise list of what the module does. It also includes model assumptions and limitations

% !TEX root = ./Basilisk-houghCircles-20190213.tex

\section{Test Description and Success Criteria}
In order to test the proper function of this module, two test images are provided.
The algorithm needs to find all the circles in the images within 1 pixel of relative error.

\section{Test Parameters}

\begin{table}[htbp]
	\caption{Error tolerance for each test.}
	\label{tab:errortol}
	\centering \fontsize{10}{10}\selectfont
	\begin{tabular}{ c | c | c  } % Column formatting, 
		\hline\hline
		\textbf{Test image}  & Expected Circles & \textbf{Tolerated Error}  \\  \hline
		mars    & 1 &1 px	   \\ 
		moons        & 6 &1 px   \\ 
		\hline\hline
	\end{tabular}
\end{table}


\section{Test Results}
The following table shows the results of the unit test described above.

\begin{table}[H]
	\caption{Test results}
	\label{tab:results}
	\centering \fontsize{10}{10}\selectfont
	\begin{tabular}{c | c  } % Column formatting, 
		\hline\hline
		\textbf{Check} 						  		&\textbf{Pass/Fail} \\ 
		\hline
	   mars   			& PASS \\ 
	   moons   			& PASS  \\
	   \hline\hline
	\end{tabular}
\end{table}

The test does not generate the result image unless called explicitly from python in order to not add images to the repository.

\begin{figure}[h!]
\centering
  \includegraphics[width = 0.5\linewidth]{../_UnitTest/result_mars.png}
  \caption{Mars Circles}
  \label{fig:mars}
\end{figure}

\begin{figure}[h!]
\centering
  \includegraphics[width = 0.5\linewidth]{../_UnitTest/result_moons.png}
  \caption{Moon crescents circles}
  \label{fig:moons}
\end{figure}

 % This includes test description, test parameters, and test results

% !TEX root = ./Basilisk-celestialTwoBodyPoint-20190311.tex

\section{User Guide}
\subsection{Input/Output Messages}
The module has 2 required input messages, 1 optional input message and 1 output message:
\begin{itemize}
	\item {\tt inputNavDataName} -- This input message, of type {\tt NavTransIntMsg}, provide the translationalnavigation states for the spacecraft.
	\item {\tt inputCelMessName} -- This input message, of type {\tt EphemerisIntMsg}, receives the first planet states for pointing
	\item {\tt inputSecMessName} -- (Optional) This input message, of type {\tt EphemerisIntMsg}, receives the second planet states for pointing
	\item {\tt outputDataName} -- This output message, of type {\tt AttRefFswMsg}, writes out the attitude, rate, and inertial derivative of the rate in order to perform control. 
\end{itemize}

\subsection{Module Parameters and States}
Outside of the message names, this module only has one other parameter:
\begin{itemize}
	\item {\tt singularityThresh} - This parameter determines the threshold after which two vectors are considered collinear. 
\end{itemize}

 % Contains a discussion of how to setup and configure  the BSK module






\bibliography{bibliography} %This includes references used and mentioned.

\end{document}
