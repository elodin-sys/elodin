% !TEX root = ./Basilisk-ThrusterForces-20160627.tex

\section{User Guide}
\begin{enumerate}


\item \textbf{$\epsilon$ Parameter}: The minimum norm inverse requires a non-zero determinant value of $[D][D]^{T}$.  For this setup, this matrix is a scalar value
\begin{equation}
	D_{2} = \text{det}([D][D]^{T})
\end{equation}
If this $D_{2}$ value is near zero, then the full 3D $\bar{\bm L}_{r}$ vector cannot be achieved.  A common example of such a scenario is with the DV thruster configuration where all $\hat{\bm g}_{t_{i}}$ axes are collinear.  Torques about these thrust axes cannot be produced.  In this case, the minimum norm solution is adjusted to only match the torques along the control matrix $[C]$ sub-space.  Torques being applied outside of $\hat{\bm c}_{j}$ is not possible as $D_{2}$ is essentially zero, indicating the other control axis cannot be controlled with this thruster configuration.  

The minimum norm torque solution is now modified to use
\begin{equation}
	\bar{\bm F} = ([C][\bar D])^{T}( [C][\bar D][\bar D]^{T}[C]^{T})^{-1} [C] {\bm L}_{r}
\end{equation}
As the thruster configuration cannot produce a general 3D torque, here the $[C]$ matrix must have either 1 or 2 control axes that are achievable with the given thruster configuration.  

To set this epsilon parameter, not the definition of the $[D]$ matrix components $\bm d_{i} = (\bm r_{i} \times \hat{\bm g}_{t_{i}})$. Note that $\bm r_{i} \times \hat{\bm g}_{t_{i}}$ is a scaled axis along which the $i^{\text{th}}$ thruster can produce a torque.  The value $\bm d_{i}$ will be near zero if the dot product of this axis with the current control axis $\hat{\bm c}_{j}$ is small.  

To determine an appropriate $\epsilon$ value, let $\alpha$ be the minimum desired angle to avoid the control axis $\hat{\bm c}_{j}$ and the scaled thruster torque axis $\bm r_{i} \times \hat{\bm g}_{t_{i}}$ being orthogonal.  If $\bar r$ is a mean distance of the thrusters to the spacecraft center of mass, then the $d_{i}$ values must satisfy
\begin{equation}
	\frac{d_{i}}{\bar r} > \cos(90\dg - \alpha) = \sin\alpha
\end{equation}
Thus, to estimate a good value of $\epsilon$, the following formula can be used
\begin{equation}
	\epsilon \approx d_{i}^{2} = \sin^{2}\!\alpha \ \bar{r}^{2}
\end{equation}
For example, if $\bar{r} = 1.3$ meters, and we want $\alpha$ to be at least 1$\dg$, then we would set $\epsilon = 0.000515$.

\item \textbf{$[C]$ matrix}: The module requires control control axis matrix $[C]$ to be defined.  Up to 3 orthogonal control axes can be selected.  Let $N_{c}$ be the number of control axes.  The $N_{c}\times 3$ $[C]$ matrix is then defined as
\begin{equation}
	[C] = \begin{bmatrix}
		\hat{\bm c}_{1}
		\\
		\vdots
	\end{bmatrix}
\end{equation}

Not that in python the matrix is given in a 1D form by defining {\tt controlAxes\_B}.  Thus, the $\hat{\bm c}_{j}$ axes are concatenated to produce the input matrix $[C]$. 

\item \textbf{\tt thrForceSign Parameter}: Before this module can be run, the parameter {\tt thrForceSign} must be set to either +1 (on-pulsing with the ACS configuration) or -1 (off-pulsing with the DV configuration).

\item \testbf{\tt use2ndLoop} Flag: If the  {\tt thrForceSign} flag is set to +1 then an on-pulsing configuration is setup.  By default the optional flag  {\tt use2ndLoop} is 0 and the algorithm  only uses the least-squares fitting loop once.  By setting the {\tt use2ndLoop} to +1 then the 2nd least squares fitting loop is used during this on-pulsing configuration,. 

\item \textbf{{\tt angErrThresh} Parameter}: The default value of {\tt angErrThresh} is 0\dg.  This means that during periods of thruster saturation the thruster force solution $\bm F$ is scaled such that $|F_{i}| \le F_{\text{max}}$.  If this scaling should only be done if $\bm\tau$ and $\bar{\bm L}_{r}$ differ by an angle $\alpha$, then set {\tt angErrThresh} equal to $\alpha$.  To turn off this force scaling during thruster saturation the parameter   {\tt angErrThresh} should be set to a value larger than 180\dg.  

\end{enumerate}