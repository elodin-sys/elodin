% !TEX root = ./Basilisk-rwMotorTorque-20190320.tex

\section{Test Description and Success Criteria}
The unit test checks for proper functionality of the module for various numbers of control axes and reaction wheel configurations, both within and outside expected bounds. The  test cases  include permutations of:
\begin{enumerate}
\item Number of control axes set to 1, 2 or 3
\item Number of RW set to 2, 4 or the maximum allowable device number
\item The optional RW availability message set to:
\begin{itemize}
\item(``NO'') -- Message not provided
\item (``ON") -- Message is provided and all RWs are set to AVAILABLE
\item (``OFF") -- Message is provided and all RWs are set to UNAVAILABLE
\item (``MIXED") -- Message is provided and half of all RWs are set to UNAVAILABLE and the remainder are AVAILABLE.
\end{itemize}
\end{enumerate}


\section{Test Parameters}

The unit test verify that the module's output reaction control torques match expectation.
\begin{table}[htbp]
	\caption{Error tolerance for each test.}
	\label{tab:errortol}
	\centering \fontsize{10}{10}\selectfont
	\begin{tabular}{ c | c } % Column formatting, 
		\hline\hline
		\textbf{Output Value Tested}  & \textbf{Tolerated Error}  \\ 
		\hline
		{\tt rwMotorTorques}        & \input{AutoTeX/toleranceValue}	   \\ 
		\hline\hline
	\end{tabular}
\end{table}




\section{Test Results}
The unit test results are shown in Table~\ref{tab:results}.  All tests should be passing.
\begin{table}[ht]
	\caption{Test results}
	\label{tab:results}
	\centering \fontsize{10}{10}\selectfont
	\begin{tabular}{c | c | c | c } % Column formatting, 
		\hline\hline
		{\tt numControlAxes} 		& {\tt numWheels}	& {\tt RWAvailMsg} 	&\textbf{Pass/Fail} \\ 
		\hline
	   1	& 2 & NO &\input{AutoTeX/passFail_12NO} \\ 
	   2	& 2 & NO &\input{AutoTeX/passFail_22NO} \\ 
	   3	& 2 & NO &\input{AutoTeX/passFail_32NO} \\ 
	   1	& 4 & NO &\input{AutoTeX/passFail_14NO} \\ 
	   2	& 4 & NO &\input{AutoTeX/passFail_24NO} \\ 
	   3	& 4 & NO &\input{AutoTeX/passFail_34NO} \\ 
	   1	& {\tt MAX\_EFF\_CNT} & NO &\input{AutoTeX/passFail_136NO} \\ 
	   2	& {\tt MAX\_EFF\_CNT}  & NO &\input{AutoTeX/passFail_236NO} \\ 
	   3	& {\tt MAX\_EFF\_CNT}  & NO &\input{AutoTeX/passFail_336NO} \\ 
	   1	& 2 & ON &\input{AutoTeX/passFail_12ON} \\ 
	   2	& 2 & ON &\input{AutoTeX/passFail_22ON} \\ 
	   3	& 2 & ON &\input{AutoTeX/passFail_32ON} \\ 
	   1	& 4 & ON &\input{AutoTeX/passFail_12ON} \\ 
	   2	& 4 & ON &\input{AutoTeX/passFail_22ON} \\ 
	   3	& 4 & ON &\input{AutoTeX/passFail_32ON} \\ 
	   1	& {\tt MAX\_EFF\_CNT} & ON &\input{AutoTeX/passFail_136ON} \\ 
	   2	& {\tt MAX\_EFF\_CNT}  & ON &\input{AutoTeX/passFail_236ON} \\ 
	   3	& {\tt MAX\_EFF\_CNT}  & ON &\input{AutoTeX/passFail_336ON} \\ 
	   1	& 2 & OFF &\input{AutoTeX/passFail_12OFF} \\ 
	   2	& 2 & OFF &\input{AutoTeX/passFail_22OFF} \\ 
	   3	& 2 & OFF &\input{AutoTeX/passFail_32OFF} \\ 
	   1	& 4 & OFF &\input{AutoTeX/passFail_14OFF} \\ 
	   2	& 4 & OFF &\input{AutoTeX/passFail_24OFF} \\ 
	   3	& 4 & OFF &\input{AutoTeX/passFail_34OFF} \\ 
	   1	& {\tt MAX\_EFF\_CNT} & OFF &\input{AutoTeX/passFail_136OFF} \\ 
	   2	& {\tt MAX\_EFF\_CNT}  & OFF &\input{AutoTeX/passFail_236OFF} \\ 
	   3	& {\tt MAX\_EFF\_CNT}  & OFF &\input{AutoTeX/passFail_336OFF} \\ 
	   1	& 2 & MIXED &\input{AutoTeX/passFail_12MIXED} \\ 
	   2	& 2 & MIXED &\input{AutoTeX/passFail_22MIXED} \\ 
	   3	& 2 & MIXED &\input{AutoTeX/passFail_32MIXED} \\ 
	   1	& 4 & MIXED &\input{AutoTeX/passFail_14MIXED} \\ 
	   2	& 4 & MIXED &\input{AutoTeX/passFail_24MIXED} \\ 
	   3	& 4 & MIXED &\input{AutoTeX/passFail_34MIXED} \\ 
	   1	& {\tt MAX\_EFF\_CNT} & MIXED &\input{AutoTeX/passFail_136MIXED} \\ 
	   2	& {\tt MAX\_EFF\_CNT}  & MIXED &\input{AutoTeX/passFail_236MIXED} \\ 
	   3	& {\tt MAX\_EFF\_CNT}  & MIXED &\input{AutoTeX/passFail_336MIXED} \\ 
	   \hline\hline
	\end{tabular}
\end{table}








