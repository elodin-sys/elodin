% !TEX root = ./Basilisk-rwMotorTorque-20190320.tex


\section{Model Functions}
The code performs the following functions:
\begin{itemize}
	\item \textbf{Maps control torque vector onto available reaction wheels}: Takes a desired body-frame torque from  \verb~CmdTorqueBodyIntMsg~ and maps it onto the RW axes. 
	\item \textbf{Removes torque from unavailable reaction wheels}: The module observes the availability of the RWs and maps the torques to only available reaction wheels. 
\end{itemize}


\section{Model Assumptions and Limitations}
This code makes the following assumptions:
\begin{itemize}
	\item The number of available wheels must be equal or larger than the number of control axes.  If this is not the case a zero output motor torque message is produced.
\end{itemize}