% !TEX root = ./Basilisk-pixelLineConverter-20190524.tex

\section{Test Description and Success Criteria}
\begin{itemize}
\item The first test runs the QR-decomposition and the back-substitution
\item The second test for this converter modules creates all three input messages. Using the same values added in those messages, it mirrors the code to output the position and covariance of the spacecraft.
\end{itemize}
\section{Test Parameters}

The unit test verify that the module output message states match expected values.
\begin{table}[htbp]
	\caption{Error tolerance for first test.}
	\label{tab:errortol}
	\centering \fontsize{10}{10}\selectfont
	\begin{tabular}{ c | c } % Column formatting, 
		\hline\hline
		\textbf{Output Value Tested}  & \textbf{Tolerated Error}  \\ 
		\hline
		QR             &  1.0E-10	   \\ 
		BackSub        & 1.0E-10   \\ 
		\hline\hline
	\end{tabular}
\end{table}

\begin{table}[htbp]
	\caption{Error tolerance for second test.}
	\label{tab:errortol}
	\centering \fontsize{10}{10}\selectfont
	\begin{tabular}{ c | c } % Column formatting, 
		\hline\hline
		\textbf{Output Value Tested}  & \textbf{Tolerated Error}  \\ 
		\hline
		{\tt r\_N}        & \input{AutoTeX/toleranceValuePos}	   \\ 
		{\tt covar\_N}        & \input{AutoTeX/toleranceValueVel}	   \\ 
		\hline\hline
	\end{tabular}
\end{table}

\section{Test Results}
The unit test is expected to pass.
\begin{table}[H]
	\caption{Test results}
	\label{tab:results}
	\centering \fontsize{10}{10}\selectfont
	\begin{tabular}{c | c  } % Column formatting, 
		\hline\hline
		\textbf{Check} &\textbf{Pass/Fail} \\ 
		\hline
	   1	   			& \input{AutoTeX/passFail} \\ 
	   2	   			& \input{AutoTeX/passFail} \\ 
	   \hline\hline
	\end{tabular}
\end{table}


